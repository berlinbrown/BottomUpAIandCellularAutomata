% %%%%%%%%%%%%%%%%%%%%%%%%%%%%%%%%%%%%%%%%%%%%%%%%%%%%%%%%%% % History
% %%%%%%%%%%%%%%%%%%%%%%%%%%%%%%%%%%%%%%%%%%%%%%%%%%%%%%%%%%

\section{History of Artificial Life}

% % 2. History of AI and Computing, Turing, VonNue, etc %    History with ALife,
% Langton, GameOfLife % %      Charles Darwin % 1937 Alan Turing % 1945 John von
% Neumann %      Wolfram % 1970 Cellular automaton, John Conway % 1986
% Christopher Langton %      Marvin Minsky

History of artificial life.  artificial intelligence, cellular automata, sixty
years ago.

Charles Darwin, Alan Turning, John von Neumann, Wolfram, John McCarthy

If we really wanted to create a thorough simulation of life, we should model
everything in our Universe.  We should go back to the big bang.  We should go
back to early earth and model the earth formations.  We should look at the first
forms of cell life.  Model water and geodesic reactions.  We should look at
everything several billion years ago and then model as much as we can from that
point and model several billion years later.

Just the thought of several billion years ago and modeling chemical reactions
and water, sun, early life.  If our goal is to model 4 billion years of earth's
history with the goal of researching complex life forms.  Just consider the
complex life on earth today.  There are 6 billion humans alive today post 4
billion years since the earth's creation.  We estimate  that 100 billion humans
have lived.

That assumes that you only consider human life to be interesting.  What about
all the other forms of interesting life forms that have lived.  How many insect
have lived since early earth.  The numbers are infestimal.

Accurately modeling ... the Universe is complex.   Modeling the earth is equally
complex.

So, we have to focus on the more interesting aspects of early life and biology.

Let's look at organic material vs inorganic

DNA, Life, Death, Mutations, Evolution.  Blind Watchmaker


Conway's Game of Life cellular automaton is one of the most prominent examples
of cellular automata theory. The one dimensional program consists of a cell grid
typically with several dozen or more rows and similar number of columns. Each
cell on the grid has an on or off Boolean state. Every cell on the grid survives
or dies to the next generation depending on the game of life rules. If there are
too many neighbors surrounding a cell then the cell dies due to overcrowding. If
there is only one neighbor cell, the base cell dies due to under-population.
Activity on a particular cell is not interesting but when you run the entire
system for many generations, a group of patterns begin to form.

You may notice some common patterns in the figure. After so many iterations
through the game of life rules, only a few cells tend to stay alive. We started
with a large random number of alive cells and over time those cells died off. In
a controlled environment you may begin with carefully placed live cells and
monitor the patterns that emerge to model some other natural phenomena.

The name Stephan Wolfram has been mentioned several times in this post. He is
the founder of Wolfram|Research, his company is known for the popular
Mathematica software suite and Wolfram|Alpha knowledge engine. He did not
initially discover cellular automata but recently he has been a prominent figure
in its advocacy. He spent 10 years working on his book, A New Kind of Science.
In the 1300 page tome, he discusses how cellular automata can be applied to
every field of science from biology to physics. NKA is a detailed study of
cellular automata programs.

The diagram above depicts the rule 30 program (or rule 30 elementary cellular
automaton). There are 8 input states (2 ^ 3) and an output state of one or zero.
If you look at the diagram from left to right. The first sequence of blocks on
the left depict an input state of { 1 1 1 } with an output of 0. Given input of
cells { 1 1 1}, the output will be set to 0. Subsequently, the next set of
blocks consist of an input state of { 1 1 0 } with an output of 0.
