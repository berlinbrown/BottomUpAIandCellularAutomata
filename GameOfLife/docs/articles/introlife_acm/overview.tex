%%%%%%%%%%%%%%%%%%%%%%%%%%%%%%%%%%%%%%%%%%%%%%%%%%%%%%%%%%%
%% Scala and Lift
%%%%%%%%%%%%%%%%%%%%%%%%%%%%%%%%%%%%%%%%%%%%%%%%%%%%%%%%%%%

\section{Introduction}

Life is all around us.  Even with inorganic material it is possible that
microscopic organisms are covering that surface.  Moving forward if we want to
study, analyze and work with artificial agents, we might consider systems that have evolved behavior over a series of steps.  We
should not necessarily build a specific tool with a specific purpose but the
creature that is built from the system may produce interesting properties which
are unlike the clean-room created software that we create today.  Most software
and hardware today is written to specification, line for line, most code written
for today's systems are created by man.  That software is designed, coded and
tested.  It would be interesting if we could start a biological like system and
interesting behavior from the system evolves over time.     

When studying life, it also makes sense to go all the back to pre-life on earth
and the creation of this planet.  When you look across the globe, there are millions of species, a
lot of life out there.  All of that life occurred from mutations, replications, chaotic collisions with other organisms. What is organic life? compared to inorganic
life?  Life evolves, minute after minute, day after day, year after year for
millions of years across a vast planet.  Inorganic matter interacts with organic matter and from within that very complex system, complex behavior emerges.  When you
look at computing systems, every piece of software, most software was created by some person.  Every line of code, every machine instruction is mostly
known to the developer or the hardware manufacturers.  The intelligent designer
in this case are teams of intelligent software and computer engineer building
complex machines to act as a tool for some purpose.  Some application.

Human beings and other animals have complex machinery embedded in them.  Animals
have eyes, ears. Human beings have the neocortex that allows us to predict
events that may occur in the future based on past memories.  But all of these
tools that animals use to survive evolved over time.  Iteration after iteration
of successes and failures.  The human brain has a purpose in the context of
human survival just like the eyes and ears have a purpose.  But do human beings
have a purpose?  Computers have a purpose.  Smart phones are tools that have a
purpose.

With software, we add all of that complex behavior.  Human beings posses amazing
skills and we use our brains to adapt to the environment.   All of that behavior
and response to the world around us was created by millions of years of mammalian evolution.  We are autonomous beings.  All of that
behavior created by seemingly random events over time.   The machinery that
makes up the human being just happen to have happened.  That is a powerful
thought.  If we can understand parts of that evolutionary process, maybe we can
model some of of those events through computer simulations and also generate
complex behavior.  Imagine if we could instantly absolute control over the
evolutionary process. Imagine if we could remove an animals brain or ears or eyes so that future
generations don't have those functions.  That animal would be helpless and
couldn't function but it just happens that they do have those capabilities and
can survive as long as their abilities allow them to.  Imagine if this force
could remove your brain so that future generations of humans didn't have a brain.  Or we removed the ability to walk, see or hear. 
Once again, we would be helpless and human progress would halt that instance.
The evolutionary process is a complex one but all of the creatures of the earth
that have survived up to this point all posses skills that allow them to
survive. The tools given them given them through that process.

\subsection{Bottom-Up versus Top-Down AI approaches}

Among them, a recent trend is to develop
commercial sensor networks that require pervasive sensing of both
environment and human beings, for example, assisted living
\cite{Akyildiz-02,Harvard-01,CROSSBOW} and smart homes
\cite{Harvard-01,Adya-01,CROSSBOW}.

% quote
\begin{quote}
``For these applications, sensor devices are incorporated into human
cloths \cite{Natarajan-01,Zhou-06,Bahl-02,Adya-01} for monitoring
health related information like EKG readings, fall detection, and voice recognition".
\end{quote}
While collecting all these multimedia information
\cite{Akyildiz-02} requires a high network throughput, off-the-shelf
sensor devices only provide very limited bandwidth in a single
channel: 19.2Kbps in MICA2 \cite{Bahl-02} and 250Kbps in MICAz.
In this article, we propose MMSN, abbreviation for Multifrequency
Media access control for wireless Sensor Networks. The main
contributions of this work can be summarized as follows.