%%--------------------------------------
%% Scala and Lift
%%--------------------------------------
\section{Object Model for Artificial Life Simulation}

I propose a model for simulating basic cell life using a bottom-up approach.  I
will look at modeling basic cells.  Adding DNA.  The cells survive and die in a
primordial pool of water, sun.  Even with this basic demo, emergent behavior
forms.


I hope have made a case that organic life is interesting.  The machinery has
already been created but we don't have a easy blueprint to recreate such complex
life, we can only use knowledge base to model as much as we can.

Here is a model for DNA. Game of Life, Cell Grid.

2D visual simulation, cells, grid, iterations or cell game cycles, cell
entities, bacteria.

Bacteria Living bacteria cell, single celled organism.

The most basic units in the life simulation consist of
Chemical Life Elements.  This enumeration contain the element types.
The element types are closely tied to the proteins on the grid
but in reality they are synonymous with chemical life elements like Carbon, Oxygen, etc.

Each grid in the life simulation consists of a chemical element unit.
The element unit contains an element level or weight.  The elements
react with other elements.

The DNA represents the CODE in this simulation.
We will decode the DNA code for protein synthesis.

For the bacteria cell, DNA can effect:

Size, weight, energy level, color, reproduction rate, 
food consumption rate, ability to handle water, sunlight/temperatures.

All forms of algae are composed of eukaryotic cells but for our demo, we are treating 
Food/Algae as a type of non organic entity.  But the cells in our system feed on this non organic form of algae.

When constructing the object model for this DNA simulation, I wanted to focus
three key components, the DNA, mutations, cell properties or traits.  But also,
there is some validation given to the water and the sunlight.  These attributes
don't effect the system as much but are part of the environment.

Model DNA bases, Adenine, Cytosine, Guanine, Thymine.

alive, processDNA, produceProteins, onStepSimulationProcessCell,
setImmutableSystemTraits, Genes, DNATranslationParser

LivingEntityCell

\subsection{Running the Simulation} 

With this basic demo, already the system exhibits interesting properties.  We
start with a few living cells.  Over time, cells are created, and cells die but
the system stabalizes with  several dozen cells.  Once mutations start to occur,
one form of entity tends to survive and the grid changes color.


Radiation destroys units of protienunit.

each cellwall_proteinchemicalelementgridunit has a bond level/strength...
  ability to withstand --- radiationprotection_level
    if radiationprotection_level < 10% then , destroyed...
    
    ....
    
    Flagellen requires energy level...   
    
    protein sytnehis only creates protein units, adds to ...
    
    
    
A cell has energy.

What is the function of the cell wall??

The cell wall will protect the cell from radiation.

If raditation too high, cell dies.

  * ProteinGridUnit Configuration determines raditation protection, chemical reaction.
  * The configuration of the ProteinGridUnit determines protection radiotion level.
  
  FUNCTION may be inhibited.
---

My flagge protein only has one function, move in water.

 -- if no flaggea proteins then you can't move.   
 
 Different cell levels.

1. Cell wall and flagella level

2. Protein, amino acid, Chemical level <---- what is in this level????
   What is in this level:  chemicals need to maintain the existence of this
                          property.

   Chemical Reaction between radiation of the sun and the protein on this cell
   
   -- Configuration -- proximity...takes energy...no energy force, dead...
   
      can a protein grid die??  no???  But more energy to do its job???
      
      configuration/count/strength//
      
  ProteinUnitGrid.


Sun can damage the DNA...

Why is water needed for life?  movement...

Viruses can inhibit growth!!!!

Bacteria CELL:
DNA FOR PROTEIN SYTHNSEISIS

 
 

