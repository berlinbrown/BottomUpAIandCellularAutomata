% %%%%%%%%%%%%%%%%%%%%%%%%%%%%%%%%%%%%%%%%%%%%%%%%%%%%%%%%%% % Artificial Life,
% Biology Computing %%%%%%%%%%%%%%%%%%%%%%%%%%%%%%%%%%%%%%%%%%%%%%%%%%%%%%%%%%

\section{Artificial Life Concepts}

Bringing together many
discpilines of science.  Computer science, neuroscience, genetics, evolutionary biology, organic chemistry.

Cover cellular automata, genetic algorithms.

Conway's Game of Life cellular automaton is one of the most prominent examples
of cellular automata theory. The one dimensional program consists of a cell grid
typically with several dozen or more rows and similar number of columns. Each
cell on the grid has an on or off Boolean state. Every cell on the grid survives
or dies to the next generation depending on the game of life rules. If there are
too many neighbors surrounding a cell then the cell dies due to overcrowding. If
there is only one neighbor cell, the base cell dies due to under-population.
Activity on a particular cell is not interesting but when you run the entire
system for many generations, a group of patterns begin to form.

You may notice some common patterns in the figure. After so many iterations
through the game of life rules, only a few cells tend to stay alive. We started
with a large random number of alive cells and over time those cells died off. In
a controlled environment you may begin with carefully placed live cells and
monitor the patterns that emerge to model some other natural phenomena.

The name Stephan Wolfram has been mentioned several times in this post. He is
the founder of Wolfram|Research, his company is known for the popular
Mathematica software suite and Wolfram|Alpha knowledge engine. He did not
initially discover cellular automata but recently he has been a prominent figure
in its advocacy. He spent 10 years working on his book, A New Kind of Science.
In the 1300 page tome, he discusses how cellular automata can be applied to
every field of science from biology to physics. NKA is a detailed study of
cellular automata programs.

Cellular automata is often used with data compression, cryptography, artificial
intelligence, urban planning, financial market modeling, music generation, and
3D terrain generation. If you are a software engineer, you may have to step back
and consider how cellular automata patterns emerge and understand the nature of
the dynamic system before looking for a typical software library. CA is not
normally seen in everyday applications. Consider this when you look at some
random pattern, don't think of the phenomenon as a random sequence of events
that cannot be replicated, think of the event in terms of a cellular automaton.
Try to imagine the rules that could model that natural behavior. Modeling
seemingly random patterns is an area where cellular automata is being widely
used. Urban planning departments are integrating geographic information systems
(GIS) with cellular automata in an attempt to predict growth in an area of a
city.
