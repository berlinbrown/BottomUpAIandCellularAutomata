%
% Equation Test and output for image export
% with gimp
\documentclass[prodmode,acmtecs]{acmsmall}

\usepackage[ruled]{algorithm2e}
\renewcommand{\algorithmcfname}{ALGORITHM}
\SetAlFnt{\small}
\SetAlCapFnt{\small}
\SetAlCapNameFnt{\small}
\SetAlCapHSkip{0pt}
\IncMargin{-\parindent}

% Metadata Information
\acmVolume{9}
\acmNumber{4}
\acmArticle{39}
\acmYear{2010}
\acmMonth{3}

% Document starts
\begin{document}

% Page heads
\markboth{G. Zhou et al.}{Equation Test}

% Title portion
\title{Equation Test}

\begin{abstract}
Data
\end{abstract}

\terms{Design, Algorithms, Performance}

\keywords{Wireless sensor networks, media access control,
multi-channel, radio interference, time synchronization}

\acmformat{Zhou, G., Wu, Y., Yan, T., He, T., Huang, C., Stankovic,
J. A., and Abdelzaher, T. F.  2010. A multifrequency MAC specially
designed for  wireless sensor network applications.}

\begin{bottomstuff}
Data
\end{bottomstuff}

\maketitle

\section{Introduction}

%%%%%%%%%%%%%%%%%%%%%%%%%%%%%

\begin{displaymath}
\int H(x,x')\psi(x')dx' = -\frac{\hbar^2}{2m}\frac{d^2}{dx^2}
                          \psi(x)+V(x)\psi(x)
\end{displaymath}

%%%%%%%%%%%%%%%%%%%%%%%%%%%%%

% Numbered Equation
\begin{equation}
\label{eqn:01}
P(t)=\frac{b^{\frac{t+1}{T+1}}-b^{\frac{t}{T+1}}}{b-1},
\end{equation}
where $t=0,{\ldots}\,,T$, and $b$ is a number greater than $1$.

according to the following equation:
% Unnumbered Equation
\[
i=\lfloor(T+1)\log_b[\alpha(b-1)+1]\rfloor.
\]
It can be easily proven that the distribution of $i$ conforms to Equation
(\ref{eqn:01}).

% Algorithm
\begin{algorithm}[t]
\SetAlgoNoLine
\KwIn{Node $\alpha$'s ID ($ID_{\alpha}$), and node $\alpha$'s
neighbors' IDs within two communication hops.}
\KwOut{The frequency number ($FreNum_{\alpha}$) node $\alpha$ gets assigned.}
$index$ = 0; $FreNum_{\alpha}$ = -1\;
\Repeat{$FreNum_{\alpha} > -1$}{
        $Rnd_{\alpha}$ = Random($ID_{\alpha}$, $index$)\;
        $Found$ = $TRUE$\;
        \For{each node $\beta$ in $\alpha$'s two communication hops
    }{
      $Rnd_{\beta}$ = Random($ID_{\beta}$, $index$)\;
      \If{($Rnd_{\alpha} < Rnd_{\beta}$) \text{or} ($Rnd_{\alpha}$ ==
          $Rnd_{\beta}$ \text{and} $ID_{\alpha} < ID_{\beta}$)\;
      }{
        $Found$ = $FALSE$; break\;
      }
        }
     \eIf{$Found$}{
           $FreNum_{\alpha}$ = $index$\;
         }{
           $index$ ++\;
     }
      }
\caption{Frequency Number Computation}
\label{alg:one}
\end{algorithm}

Bus masters are divided into two disjoint sets, $\mathcal{M}_{RT}$
and $\mathcal{M}_{NRT}$.
% description
\begin{description}
\item[RT Masters]
$\mathcal{M}_{RT}=\{ \vec{m}_{1},\dots,\vec{m}_{n}\}$ denotes the
$n$ RT masters issuing real-time constrained requests. To model the
current request issued by an $\vec{m}_{i}$ in $\mathcal{M}_{RT}$,
three parameters---the recurrence time $(r_i)$, the service cycle
$(c_i)$, and the relative deadline $(d_i)$---are used, with their
relationships.
\item[NRT Masters]
$\mathcal{M}_{NRT}=\{ \vec{m}_{n+1},\dots,\vec{m}_{n+m}\}$ is a set
of $m$ masters issuing nonreal-time constrained requests. In our
model, each $\vec{m}_{j}$ in $\mathcal{M}_{NRT}$ needs only one
parameter, the service cycle, to model the current request it
issues.
\end{description}

\section{Simulator}
\label{sec:sim}

% enumerate
\begin{enumerate}
\item Load state into microcontroller model.
\item For each assignment.
      \begin{enumerate}
      \item either call interrupt handler or simulate effect of next instruction, or
      \item evaluate truth values of atomic propositions.
      \end{enumerate}
\item Return resulting states.
\end{enumerate}

% Figure

\subsection{Problem Formulation}

% Enunciations

\begin{eqnarray}%
s=\frac{10}{N_{1}}-\frac{10}{N_{2}}. \nonumber
\end{eqnarray}%

\end{document}

